\section*{Ejercicio 2}

Sean $V$ un $F$-espacio vectorial de dimensión finita, $W$ un subespacio vectorial de $V$ y $T: V \rightarrow V$
una proyección sobre $W$. Escoger una base ordenada adecuada de $V$ tal que $\left[ T \right]_{\beta}$
sea la matriz diagonal. \\

\noindent \solucion \\

\textbf{Proyección sobre un subespacio vectorial:} Una proyección sobre
un subespacio vectorial $W$ es una transformación lineal 
$T: V \rightarrow V$ tal que para cualquier vector $v$ en $V$,
$T(v)$ es el punto más cercano a $v$ que está en el subespacio $W$. 
En otras palabras, $T(v)$ es el vector en $W$ que está más cerca de $v$.

\textbf{Matriz de transformación lineal:} Cualquier transformación lineal
$T: V \rightarrow V$ puede representarse mediante una matriz.
En particular, si tenemos una base ordenada $\beta$ de $V$,
la matriz de $T$ respecto a $\beta$, denotada como $[T]_{\beta}$,
es la matriz que describe cómo $T$ actúa sobre los vectores en 
$V$ cuando se expresan en términos de la base $\beta$.

\subsubsection*{Paso 1: Encontrar una base ordenada para $W$}

Supongamos que para $\beta_W = \{w_1, w_2, \ldots, w_k\}$ es base de $W$. Sea la cantidad de vectores que pueda tener forsozamente deben ser linealmente independiente y generar $W$.

\subsubsection*{Paso 2: Ampliar $\beta_W$ a una base ordenada de $V$}

Tomamos los vectores de $\beta_W$ para formar una base ordenada de $V$. Supongamos que 
\begin{equation*}
    \beta = \{w_1, w_2, \ldots, w_k, v_{k+1}, v_{k+2}, \ldots, v_n\}
\end{equation*}
es una base para $V$, donde $n$ es la dimensión de $V$. Esta base tiene $k$ vectores de $W$ y $(n - k)$ vectores adicionales que completan la base de $V$.


\subsubsection*{Paso 3: Proyección $T$ en términos de la base $\beta$}

Para cualquier vector $v$ en $V$, la proyección $T(v)$ es igual a $v$ si $v$ está en $W$, y es igual a $\mathbf{0}$ si $v$ está en el complemento ortogonal de $W$. Podemos expresar $T(v)$ en términos de la base $\beta$ de la siguiente manera:

\[
T(v) = \begin{cases} v, & \text{si } v \in W \\ \mathbf{0}, & \text{si } v \in \text{complemento ortogonal de } W \end{cases}
\]

\subsubsection*{Paso 4: Matriz $[T]_{\beta}$}

La matriz $[T]_{\beta}$ tendrá una forma diagonal, donde los bloques correspondientes a $W$ serán matrices identidad y los bloques correspondientes al complemento ortogonal de $W$ serán matrices nulas. Es decir,

\[
[T]_{\beta} = \begin{pmatrix} I_k & 0 \\ 0 & 0 \end{pmatrix}
\]

donde $I_k$ es la matriz identidad de tamaño $k \times k$, y $0$ representa una matriz nula de tamaño $(n - k) \times (n - k)$.

Ahora vamos a considerar el espacio vectorial $V = \mathbb{R}^3$ con la proyección $T: \mathbb{R}^3 \rightarrow \mathbb{R}^3$ sobre el subespacio $W$ generado por $(1, 0, 0)$.

\begin{itemize}
    \item $\beta_W = \{(1, 0, 0)\}$ es una base para $W$.
    \item Ampliando $\beta_W$, obtenemos la base $\beta = \{(1, 0, 0), (0, 1, 0), (0, 0, 1)\}$ para $V$.
\end{itemize}

La matriz $[T]_{\beta}$ en esta base es:

\[
[T]_{\beta} = \begin{pmatrix} 1 & 0 & 0 \\ 0 & 0 & 0 \\ 0 & 0 & 0 \end{pmatrix}
\]

Para demostrar que $[T]_{\beta}$ tiene una forma diagonal, consideremos cómo actúa $T$ sobre los vectores en $V$ expresados en la base $\beta$. Para cualquier $v$ en $V$, $T(v)$ tiene dos posibles casos:

\begin{enumerate}
    \item Si $v$ está en $W$, entonces $T(v) = v$. En términos de la matriz $[T]_{\beta}$, esto corresponde a multiplicar $[T]_{\beta}$ por $v$ y obtener $v$. Esto se logra usando el bloque $I_k$ en la esquina superior izquierda de $[T]_{\beta}$.

    \item Si $v$ está en el complemento ortogonal de $W$, entonces $T(v) = \mathbf{0}$. En términos de la matriz $[T]_{\beta}$, esto corresponde a multiplicar $[T]_{\beta}$ por $v$ y obtener el vector nulo. Esto se logra usando los bloques de $0$ en la parte inferior de $[T]_{\beta}$.

\end{enumerate}

Por lo tanto, la matriz $[T]_{\beta}$ tiene una forma diagonal como se muestra en el paso 4.