\section*{Ejercicio 1}

Sean $X$ un conjunto no vacío y $F$ un campo.
Sea $f[X] \coloneqq \{f : f \colon X\to F\}$ el conjunto de todas las funciones que van de $X$ a $F$. Entonces $f[x]$ es un espacio vectorial sobre $F$ con las siguientes operaciones para cualesquiera $x \in X$ y $a \in F$:

$$(f+g)(x) = f(x) + g(x)$$

$$(af)(x) = a f(x)$$


\noindent Demuéstralo o escribe cuál es la propiedad que no cumple y justifica tu respuesta
\begin{enumerate}
    \item Cerradura bajo la suma: 
    
    Para demostrar que la suma de dos elementos en $f[x]$ sigue estando en $f[x]$, tomemos dos funciones arbitrarias $f$ y $g$ en $f[x]$. Entonces, para cualquier $x$ en $X$, tenemos:
    
    \begin{align*}
        (f + g)(x) &= f(x) + g(x) \\
        &\in F + F \quad \text{(porque } f(x) \text{ y } g(x) \text{ están en } F\text{)} \\
        &\subseteq F \quad \text{(ya que } F \text{ es cerrado bajo la suma)} \\
    \end{align*}
    
    Por lo tanto, $(f + g)(x)$ está en $F$ para todo $x$, lo que significa que la suma $f + g$ es una función de $X$ a $F$ y, por lo tanto, pertenece a $f[x]$. La cerradura bajo la suma se cumple.

    \item Cerradura bajo la multiplicación por escalar: Para demostrar la cerradura bajo la multiplicación por escalar en $f[x]$, consideremos una función $f \in f[x]$ y un escalar $a \in F$. La multiplicación por escalar se define como:
    
    \begin{align*}
        (af)(x) &= a f(x) \in F \quad \text{(Definición de multiplicación por escalar)}
    \end{align*}
    
    Dado que $f(x)$ está en $F$ (ya que $f \in f[x]$) y $a$ también está en $F$, el producto $af(x)$ también estará en $F$. Esto significa que $(af)(x)$ está en $F$ para todo $x$, y por lo tanto, la función $af(x)$ pertenece a $f[x]$. La cerradura bajo la multiplicación por escalar se cumple.

    \item Asociatividad de la suma: 
    
    La propiedad de asociatividad de la suma en $f[x]$ sigue directamente de la asociatividad de la suma en $F$, ya que estamos sumando funciones de $X$ a $F$ y, en cada punto $x$, estamos sumando elementos de $F$. Entonces para cualesquiera $f, g, h \in f[x]$ y $x \in X$, tenemos:
    
    \begin{align*}
        [(f + g) + h](x) &= (f + g)(x) + h(x) \quad \text{(Definición de suma en } f[x]\text{)} \\
        &= [f(x) + g(x)] + h(x) \quad \text{(Definición de suma en } f[x]\text{)} \\
        &= f(x) + [g(x) + h(x)] \quad \text{(Asociatividad en } F\text{)} \\
        &= f(x) + (g + h)(x) \quad \text{(Definición de suma en } f[x]\text{)} \\
        &= [(f + (g + h))(x) \quad \text{(Definición de suma en } f[x]\text{)}
    \end{align*}
    
    Como esta igualdad es válida para todo $x \in X$, concluimos que $(f + g) + h = f + (g + h)$ en $f[x]$. La propiedad de asociatividad se cumple.

    \item Conmutatividad de la suma: 
    
    La propiedad de conmutatividad de la suma en $f[x]$ se deriva de la conmutatividad de la suma en el campo $F$. Esto se debe a que, en cada punto $x \in X$, estamos sumando elementos del campo $F$, y estas sumas individuales se comportan de acuerdo con la conmutatividad de $F$. Formalmente, para cualesquiera $f, g \in f[x]$ y $x \in X$, tenemos:
    
    \begin{align*}
        (f + g)(x) &= f(x) + g(x) \quad \text{(Definición de suma en } f[x]\text{)} \\
        &= g(x) + f(x) \quad \text{(Conmutatividad en } F\text{)} \\
        &= (g + f)(x) \quad \text{(Definición de suma en } f[x]\text{)}
    \end{align*}
    
    Como esta igualdad es válida para todo $x \in X$, concluimos que $f + g = g + f$ en $f[x]$. La propiedad de conmutatividad se cumple.

    \item Elemento neutro de la suma: 
    
    La función nula o cero, que mapea todos los elementos de $X$ a $0$ en $F$, actúa como el elemento neutro de la suma en $f[x]$. Formalmente, para cualquier función $f \in f[x]$ y cualquier $x \in X$, tenemos:
    
    \begin{align*}
        (0 + f)(x) &= 0(x) + f(x) \quad \text{(Definición de suma en } f[x]\text{)} \\
        &= 0 + f(x) \quad \text{(Definición de la función nula)} \\
        &= f(x) \quad \text{(Propiedad del elemento neutro en } F\text{)}
    \end{align*}
    
    Como $0$ es la función nula en $f[x]$ y esta igualdad se cumple para todo $x \in X$, concluimos que $0 + f = f$ para toda $f \in f[x]$. La función nula actúa como el elemento neutro de la suma.

    \item Inverso aditivo:
    
    Dada una función $f$ en $f[x]$, su inverso aditivo sería la función $-f(x)$, que mapea cada $x$ a $-f(x)$ en $F$. Esta función existe en $f[x]$ debido a la estructura del campo $F$. Formalmente, para cualquier función $f \in f[x]$ y cualquier $x \in X$, tenemos:
    
    \begin{align*}
        (f + (-f))(x) &= f(x) + (-f)(x) \quad \text{(Definición de suma en } f[x]\text{)} \\
        &= f(x) - f(x) \quad \text{(Definición de } -f(x)\text{)} \\
        &= 0 \quad \text{(Propiedad de inversos aditivos en } F\text{)}
    \end{align*}
    
    Como esta igualdad se cumple para todo $x \in X$, concluimos que $f + (-f) = 0$ para toda $f \in f[x]$. El inverso aditivo existe y es la función $-f(x)$.
    
    \item Distributividad de la suma de escalares sobre la suma de funciones: 
    
    Dados $a \in F$ y funciones $f, g \in f[x]$, tenemos
    
    \begin{align*}
        a(f + g)(x) &= a(f(x) + g(x)) \\
        &= af(x) + ag(x) \\
        &= (af)(x) + (ag)(x) \\
        &= (af + ag)(x)
    \end{align*}
    
    Por lo tanto, la distributividad se cumple.

    \item Distributividad de la suma de escalares sobre la multiplicación de funciones: 
    
    Dados $a, b \in F$ y una función $f \in f[x]$, tenemos
    
    \begin{align*}
        (a + b)f(x) &= af(x) + bf(x) \\
        &= (af)(x) + (bf)(x) \\
        &= (af + bf)(x)
    \end{align*}
    
    Por lo tanto, la distributividad se cumple.
    
\end{enumerate}

Todas las propiedades necesarias para que $f[x]$ sea un espacio vectorial sobre $F$ están satisfechas. Por lo tanto, las operaciones definidas cumplen con las condiciones requeridas para formar un espacio vectorial de funciones.
