\section*{Ejercicio 2}

En cada uno de los siguientes incisos, determine (y demuestre, si es el caso) si el conjunto es linealmente
independiente o no.

\begin{itemize}
    \item[(a)] En $P_{2}(\mathbb{R})$ el conjunto $S = \{ 1, 1 - x + 2x^{2}, 2 + 3x - x^{2} \}$  

    Para comprobar si el conjunto \(S = \{1, 1 - x + 2x^{2}, 2 + 3x - x^{2}\}\) en \(P_{2}(\mathbb{R})\) 
    es linealmente independiente, primero,  vemos la definición de \textit{linealmente independiente},
    donde para un conjunto de polinomios 
    \(\{p_1(x), p_2(x), p_3(x), \ldots, p_n(x)\}\)
    se considera linealmente independiente en \(P_{n}(\mathbb{R})\)
    si la única forma de escribir la combinación lineal 
    igual a cero es haciendo que todos los coeficientes
    sean iguales a cero:

    $$ c_{1} p_{1} (x) + c_{2}p_{2}(x) + c_{3} p_{3} (x) + \cdots + c_{n} p_{n}(x)$$

    donde \(c_1, c_2, c_3, \ldots, c_n\) son coeficientes reales y \(p_i(x)\) son los polinomios en el conjunto.

    Ahora la combinación lineal la igualamos a cero. 

    \[
        c_1(1) + c_2(1 - x + 2x^{2}) + c_3(2 + 3x - x^{2}) = 0
    \]

    Donde \(c_1, c_2, c_3\) son los coeficientes que queremos 
    encontrar. Distribuimos los coeficientes en la ecuación
    y agrupamos términos semejantes:

    \[
        c_1 + c_2(1 - x + 2x^{2}) + c_3(2 + 3x - x^{2}) = 0
    \]

    Obtenemos:    
    \begin{align*}
        c_1 + c_2 - c_2x + 2c_2x^{2} + 2c_3 + 3c_3x - c_3x^{2} & = 0 \\
        (c_1 + c_2 + 2c_3) + (-c_2 - c_3)x + (2c_2 - c_3)x^{2} & = 0
    \end{align*}
    

    Entonces la ecuación debe ser igual a cero para todo \(x\), y como esta ecuación debe ser verdadera para todos los valores de \(x\), los coeficientes de cada término deben ser igual a cero. Esto nos lleva a un sistema de ecuaciones:

    \begin{align*}
        1. & \quad c_1 + c_2 + 2c_3 = 0 \\
        2. & \quad -c_2 - c_3 = 0 \\
        3. & \quad 2c_2 - c_3 = 0
    \end{align*}

    De la ecuación 2, tenemos:
    \begin{align*}
        -c_2 - c_3 = 0 & \\        
        \implies &  c_3 = -c_2
    \end{align*}

    Sustituimos este valor en la ecuación 3:
    \begin{align*}
        2c_2 - c_3    & = 0  \\
        2c_2 - (-c_2) & = 0  \\
        2c_2 + c_2    & = 0  \\
        3c_2          & = 0  \\
    \end{align*}


    Por lo tanto, \(c_2 = 0\). Ahora, sustituimos \(c_2 = 0\) en la ecuación 1:
    \begin{align*}
        c_1 + c_2 + 2c_3 = 0  & = 0  \\
        c_1 + 0 + 2(-c_2) = 0 & = 0  \\
        c_1 - 2c_2 = 0        & = 0  \\
        c_1 - 2(0) = 0        & = 0  \\
        c_1                   & = 0
    \end{align*}

    Hemos encontrado que \(c_1 = 0\), \(c_2 = 0\), y \(c_3 = 0\),
    lo que significa que la única forma de obtener la combinación
    lineal igual a cero es haciendo que todos los coeficientes 
    sean iguales a cero. Por lo tanto, el conjunto
     \(S = \{1, 1 - x + 2x^{2}, 2 + 3x - x^{2}\}\) 
     es linealmente independiente en \(P_{2}(\mathbb{R})\).

    %%%%%%%%%%%%%%%%%%%%%%%%%%%%%%%%%%%%%%%%%%%%%%%%%%%%%%%%%%%%%%%%%%%%%%%%
    \item[(b)] En $ M_{3 \times 3}$, el conjunto
    \begin{equation}
        S = \left\{
            \begin{pmatrix}
                6 & 2 \\
                -5 & 0 \\
            \end{pmatrix},         
            \begin{pmatrix}
                0 & 3 \\
                -4 & -5 \\
            \end{pmatrix},        
            \begin{pmatrix}
                3 & 0 \\
                -2 & 5 \\
            \end{pmatrix}
            \right\}
    \end{equation}







    
\end{itemize}