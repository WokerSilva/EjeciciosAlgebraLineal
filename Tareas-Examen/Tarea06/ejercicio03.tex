\section*{Ejercicio 3}

Sea $T: \R{3} \rightarrow \R{3}$ dada por $T(x,y,z) = (x, 2y + x, z)$. 
\begin{enumerate}
    \item Demuestre que $T$ es una transformación lineal inyectiva y suprayectiva

    Veamos que para que T=$\overline{0}$, se tiene que cumplir el siguiente sistema de ecuaciones:

    \begin{align*}
        x = & 0 \\
        2y + x =& 0 \\ 
        z = & 0 
    \end{align*}

    \begin{equation*}
       x=0 \Longrightarrow 2y+0=0 \Longrightarrow y=0/2=0
   \end{equation*}
   \begin{equation*}
       T(x,y,z)=(0,0,0) \Longrightarrow x=y=z=0
   \end{equation*}
   Tenemos que T:$\R^3 \rightarrow \R^3$, por el Teorema 1.22, T es inyectiva, y dado a que ambos espacios son el mismo, su dimensión coincide, por lo que, por el Teorema 1.25, T es suprayectiva.\\


    %%%%%%%%%%%%%%%%%%%%%%%%%%%%%%%%%%%%%%%%%%%%%%%%%%%%%%%%%%%%%%%%%%%%%%%%%%%%%%%%%%%%%%%%%%%%%%%%%%%%%%%%%%%%%%%%%%%%%%%%%%%%%%%%
    \item Calcular $\left[ T \right]_{\beta}^{\gamma}$ y $\left( \left[ T \right]_{\beta}^{\gamma} \right)^{-1}$
        \begin{equation*}
            \beta = \{ (1,0,0), (1,2,0), (0,0,1) \}
        \end{equation*}
    
        y 
    
        \begin{equation*}
            \gamma = \{ (0,0,1), (3,0,0), (0,1,0) \}
        \end{equation*}
        
        \vspace{10mm}

        Necesitamos encontrar los escalares que cumplen con la definición de vectores de representación para cada T de $\beta$:\\
\\
T((1, 0, 0))=[(1), 2(0)+(1), (0)]=(1,1,0)=\_(0, 0, 1)+\_(3, 0, 0)+\_(0, 1, 0)\\
\\
T((1, 2, 0))=[(1), 2(2)+(1), (0)]=(1,5,0)=\_(0, 0, 1)+\_(3, 0, 0)+\_(0, 1, 0)\\
\\
T((0, 0, 1))=[(0), 2(0)+(0), (1)]=(0,0,1)=\_(0, 0, 1)+\_(3, 0, 0)+\_(0, 1, 0)\\
\\
Tenemos que:\\
(1,1,0)=0(0, 0, 1)+1/3(3, 0, 0)+1(0, 1, 0)\\
\\
(1,5,0)=0(0, 0, 1)+1/3(3, 0, 0)+5(0, 1, 0)\\
\\
(0,0,1)=1(0, 0, 1)+0(3, 0, 0)+0(0, 1, 0)\\
\\
Por lo tanto, la matriz de representación $[T]^{\gamma}_{\beta}$ es la siguiente:
\begin{equation*}
[T]^{\gamma}_{\beta}= 
\begin{pmatrix}
    0 & 0 & 1\\
    1/3 & 1/3 & 0\\
    1 & 5 & 0\\
\end{pmatrix}
\end{equation*}
   Ahora, calculamos la inversa de $[T]^{\gamma}_{\beta}$:
   \begin{equation*}
       ([T]^{\gamma}_{\beta})^{-1}=\frac{1}{det[T]^{\gamma}_{\beta}}Adj([T]^{\gamma}_{\beta})
   \end{equation*}
   \begin{equation*} 
   Adj([T]^{\gamma}_{\beta})=
      \begin{pmatrix}
  0	& 5	&-1/3\\
  0 &	-1	& 1/3\\
4/3	& 0	 &  0\\
      \end{pmatrix} 
   \end{equation*}
   \begin{equation*}
     Det[T]^{\gamma}_{\beta})=1(5(1/3)-1(1/3))=4/3
   \end{equation*}
   \begin{equation*}\frac{1}{4/3}Adj([T]^{\gamma}_{\beta})= \frac{3}{4}
       \begin{pmatrix}
            0	& 5	&-1/3\\
  0 &	-1	& 1/3\\
4/3	& 0	 &  0\\ 
       \end{pmatrix}
   \end{equation*}
   \begin{equation*}
   ([T]^{\gamma}_{\beta})^{-1}=
       \begin{pmatrix}
0	& 15/4	& -1/4\\
0	& -3/4	&  1/4\\
1	&    0	&    0\\

       \end{pmatrix}
   \end{equation*}




    %%%%%%%%%%%%%%%%%%%%%%%%%%%%%%%%%%%%%%%%%%%%%%%%%%%%%%%%%%%%%%%%%%%%%%%%%%%%%%%%%%%%%%%%%%%%%%%%%%%%%%%%%%%%%%%%%%%%%%%%%%%%%%%%%%%
    \item Calcule $T^{-1}$  y verifique que $\left( \left[ T \right]_{\beta}^{\gamma} \right)^{-1}  = \left[ T \right]_{\gamma}^{\beta}$ 
    
    Dado a que T es biyectiva, es invertible, por lo que podemos calcular $T^{-1}$:\\
\\
$T^{-1}$(T(x,y,z))=(x,y,z)\\
\\
$T^{-1}$((x, 2y+x, z))=(x,y,z)
\begin{equation*}
    x=x
\end{equation*}
\begin{equation*}
    2y-x \Longrightarrow y=\frac{y-x}{2}
\end{equation*}
\begin{equation*}
z=z
\end{equation*}
\begin{equation*}
    T^{-1}=(x,\frac{y-x}{2},z)
\end{equation*}
Hacemos la representación matricial en torno a $\gamma$ y $\beta$
\\
$T^{-1}$((0, 0, 1))=[(0), $\frac{(0)-(0)}{2}$, (1)]=(0, 0, 1)=\_(1, 0, 0)+\_(1, 2, 0)+\_(0, 0, 1)\\
\\
$T^{-1}$((3, 0, 0))=[(3), $\frac{(0)-(3)}{2}$, (0)]=(3,$-\frac{3}{2}$,0)=\_(1, 0, 0)+\_(1, 2, 0)+\_(0, 0, 1)\\
\\
$T^{-1}$((0, 1, 0))=[(0), $\frac{(1)-(0)}{2}$, (1)]=(0,$\frac{1}{2}$,0)=\_(1, 0, 0)+\_(1, 2, 0)+\_(0, 0, 1)\\
\\
\\
Tenemos entonces que:\\
(0, 0, 1)=0(1, 0, 0)+0(1, 2, 0)+1(0, 0, 1)\\
\\
(3,$-\frac{3}{2}$,0)=$\frac{15}{4}$(1, 0, 0)+$-\frac{3}{4}$(1, 2, 0)+(0, 0, 1)\\
\\
(0,$\frac{1}{2}$,0)=$-\frac{1}{4}$ (1, 0, 0)+ $\frac{1}{4}$ (1, 2, 0)+0(0, 0, 1)\\
\\
Tenemos que la matriz de representación de la inversa de T es:
\begin{equation*}
    [T^{-1}]^{\beta}_{\gamma} =
    \begin{pmatrix}
        0	& 15/4	& -1/4\\
0	& -3/4	&  1/4\\
1	&    0	&    0\\
    \end{pmatrix}
\end{equation*}
Tenemos entonces que:
\begin{equation*}
     [T^{-1}]^{\beta}_{\gamma}=([T]^{\gamma}_{\beta})^{-1}
\end{equation*}

\end{enumerate}

