\section*{Ejercicio 3}

¿Las siguientes transformaciones son inyectivas, suprayectivas o ambas?

\begin{enumerate}
    \item $T: \R^{2} \rightarrow \R^{3}$ tal que $T(a,b) = (a + b, 0, 2a - b)$
    



%%%%%%%%%%%%%%%%%%%%%%%%%%%%%%%%%
%%%%%%%%%%%%%%%%%%%%%%%%%%%%%%%%%
%%%%% EJERCICIO 3.B
    \item $T: \R^{5} \rightarrow \R^{5}$ tal que $T(a,b,c,d,e)  = (a + 2b -c, -3a + b + 4c, a - b + 2d, b + c +3e, 2a + b + d - e)$\\
%%%%%%%%%%%%%%%%%%%%%%%%%%%%%%%%%


      Para demostrar que la transformación lineal $T: \R^{5} \rightarrow \R^{5}$ es inyectiva, debemos demostrar
      que el núcleo (kernel) de $T$ es el vector nulo (0,0,0,0,0). Esto significa que si $T(a,b,c,d,e) = T(x,y,z,w,v)$,
      entonces $a = x, b = y, c = z, d = w, e = v$. En otras palabras $T(a,b,c,d,e) = T(x,y,z,w,v)$ implica que 
      $(a,b,c,d,e) = (x,y,z,w,v)$\\

      La transformación lineal $T$ se define como:

      \begin{equation*}
            T(a,b,c,d,e)  = (a + 2b -c, -3a + b + 4c, a - b + 2d, b + c +3e, 2a + b + d - e)
      \end{equation*}

      Para demostrar la inyectividad, supongamos que $T(a,b,c,d,e) = T(x,y,z,w,v)$, lo que significa que:

      \begin{equation*}
            (a + 2b -c, -3a + b + 4c, a - b + 2d, b + c + 3e, 2a + b + d - e) = T(x,y,z,w,v) 
      \end{equation*}

      Esto nos lleva a uns sitema de ecuaciones lineales: 
      \begin{equation}
            a + 2b -c = x \tag{3.1} \label{ec3.1}
      \end{equation}
      \begin{equation*}
            -3a + b + 4c = y \tag{3.2} \label{ec3.2}
      \end{equation*}
      \begin{equation}
            a - b + 2d = z \tag{3.3} \label{ec3.3}
      \end{equation}
      \begin{equation}
            b + c + 3e = w \tag{3.4} \label{ec3.4}
      \end{equation}
      \begin{equation}
            2a + b + d - e = v \tag{3.5} \label{ec3.5} 
      \end{equation}

      Queremos demostrar que la única solución para este sistema es $(a,b,c,d,e) = (x,y,z,w,v)$. Para
      hacerlo, podemos utilizar álgebra lineal y métodos de resolución de sistemas de ecuaciones

      Conocemos la primera ecuación \eqref{ec3.1}
            
      Restamos la primera ecuación \eqref{ec3.1} de la segunda \eqref{ec3.2}

      \begin{equation*}
            (-3a + b + 4c) - (a + 2b -c) = y -x
      \end{equation*}
      
      Simplificamos 
      \begin{equation*}
            -2a - b + 5c = y - x
      \end{equation*}
      
      Despejamos $a$
      \begin{equation*}
            a = \frac{y-x+b+5c}{-2}
      \end{equation*}
      Ahora, utilicemos esta expresión para $a$ en la tercera ecuación \eqref{ec3.3}
      \begin{equation*}
            \left( \frac{y-x+b+5c}{(-2)}   \right)  - b + 2d = z
      \end{equation*}
      Simplifiquemos:
      \begin{equation*}
            (-y + x -b - 5c) -3b +4d = 2z
      \end{equation*}
      Despejamos $b$
      \begin{equation*}
            b = \frac{x-y-5c-2z}{3} + 2d
      \end{equation*}
      Ahora sustituyamos esta expresión para $b$ en la cuarta ecuación \eqref{ec3.4}
      \begin{equation*}
            \left( \frac{x - y - 5c  - 2}{3} \right) + c  + 3e = w
      \end{equation*}
      Simplificamos
      \begin{equation*}
            \frac{x-y-2z}{3} + 2d +4c +3e = w
      \end{equation*}
      Despejamos $c$ 
      \begin{equation*}
            c = \frac{(3w - x + y 2z - 6d -9e)}{4}
      \end{equation*}
      Finalmente, utilicemos estas expresiónes para $a$, $b$ y $c$ en la quinta ecuación \eqref{ec3.5}
      \begin{equation*}
            2 \left(\frac{y-x+b+5c}{2}\right) + \left(\frac{x-y-5c-2z}{3+2d-e}\right) = v
      \end{equation*}
      Simplifiquemos
      \begin{align*}
            -(y-x+b+5c) + \frac{x-y-5c -2z}{3 + 2d -e} & \,\, = v \\
            -\left(y-x+ (\frac{x-y-5c-2z}{3}) + 5c\right) + 2d -3 & \,\, = v  \\
            -\left(y-x+\frac{x}{3} - \frac{y}{3} - \frac{5c}{3} - \frac{2z}{3} + 5c\right) + 2d -e & \,\, = v \\
            -\frac{2y}{3} + \frac{2x}{3} -\frac{7c}{3} - \frac{2z}{3}  + 2d - e & \,\, = v \\
            \frac{(2x -2y -7c -2z +6d -3e)}{3} & \,\, = v
      \end{align*}
      De esta ecuación, podemos despejar $e$
      \begin{equation*}
            e = \frac{(2x - 2y -7c - 2z +6d -3v)}{3}
      \end{equation*}

      Hemos expresado tadas las variables $a,b,c,d,e$ en terminos de $(x.y,z,w,v)$. Esto demuestra que para cualquier 
      conjunto de varoles $(x.y,z,w,v)$ en $\R^{5}$ existe un conjunto correspondiente de valroes $(a,b,c,d,e)$ en $\R^{5}$
      que satisface el sistema de ecuaciones. $\therefore$ Hemos demostrado que $T$ es inyectiva.

\end{enumerate}























