\section*{Ejercicio 1}

Determina si los siguientes conjuntos son base para $P_2(\mathbb{R})$:

\begin{enumerate}
    %%%%%%%%%%%%%%%%%%%%%%%%%%%%%%%%%%%%%%%%%%%%%%%%%%%%%
    %%%%%%%%%%%%%%%%%%%%%%%%%%%%%%%%%%%%%%%%%%%%%%
    %%%% Ejercicio 1.1
    %%%%%%%%%%%%%%%%%%%%%%%%%%%%%%%%%%%%%%%%%%%%%%
    %%%%%%%%%%%%%%%%%%%%%%%%%%%%%%%%%%%%%%%%%%%%%%%%%%%%%
    \item $\{ 1 - x + 2x^{2},\, 2 + x - 2x^{2},\, 1 - 2x + 4x^{2} \}$
    
    Para verificar si el conjunto $\{1 - x + 2x, 2 + x - 2x, 1 - 2x + 4x\}$ es una base para el espacio vectorial $\mathbb{P}_2$, debemos comprobar dos cosas:
    \begin{enumerate}
        \item[1] \textbf{¿El conjunto es linealmente independiente?} Para contestar esto comprobaremos
        si los 3 polinomios en el conjunto son linealmente independientes, para ello la combinación
        lineal de los 3 polinomios tiene como resultado el polinomio cero. Sean $a, b$ y $c$
        coeficientes escalares tenemos que comprobar que $a = b = c = 0$ en la ecuación:

        $$a(1 - x + 2x^2) + b(2 + x - 2x^2) + c(1 - 2x + 4x^2)$$

        Desarrollamos
        \begin{align*}
            a(1 - x + 2x^2) + b(2 + x - 2x^2) + c(1 - 2x + 4x^2) &= 0 \\
            a - ax + 2ax^2 + 2b + bx - 2bx^2 + c - 2cx + 4cx^2 &= 0 \\
            (a + 2b + c) + (-ax + bx - 2cx) + (2ax^2 - 2bx^2 + 4cx^2) &= 0 \\
            (a + 2b + c) + (-a + b - 2c)x + (2a - 2b + 4c)x^2 &= 0 \\
        \end{align*}
            
        Obteniendo así el siguiente sistema de ecuaciones:

        \begin{equation}
            a + 2b + c = 0                
        \end{equation}
        \begin{equation}
            -a + b -2c = 0
        \end{equation}
        \begin{equation}
            2a - 2b + 4c = 0
        \end{equation}

        Para resolverlo, despejamos a de la ecuación $(1)$ : $a = -2b - c$. \\
        Sustituimos $a$ en $(2)$ y $(3)$ para tener:

        \begin{equation}
            -(-2b - c) +b -2c = 0 \tag{2}
        \end{equation}
        \begin{equation}
            2(-2b -c) -2b +4c = 0 \tag{3}
        \end{equation}
        
        Simplificamos y agrupamos términos semejantes:
        \begin{equation}
            2b + c + b -2c = 0 \tag{2}
        \end{equation}
        \begin{equation}
            -4b -2c - 2b + 4c = 0 \tag{3}
        \end{equation}
        \begin{center}
            $\downarrow$
        \end{center}
        \begin{equation}
            3b - c  = 0 \tag{2}
        \end{equation}
        \begin{equation}
            -6b + 2c = 0 \tag{3}
        \end{equation}

        Ahora tenemos un sistema de 2 ecuaciones con incógnitas $b$ y $c$. Despejamos $c$ de $(2)$
        y tenemos $c = 3b$, sustituimos $c$ en $3$ y tenemos $-6b + 2 (3b) = 0$, despejamos $b$ y obtenemos:

        \begin{align*}
            -6b + 2(3b) & = 0 \\
            -6b + 6b    & = 0 \\
            0 & =  \\
            b & = ?
        \end{align*}                
    \end{enumerate}
    Lo que significa que $b$ puede ser cualquier número real. Y como $b \neq 0$, el conjunto no 
    cumple con ser linealmente independiente y, por lo tanto, no es una base para $P_2(\mathbb{R})$

    %%%%%%%%%%%%%%%%%%%%%%%%%%%%%%%%%%%%%%%%%%%%%%%%%%%%%
    %%%%%%%%%%%%%%%%%%%%%%%%%%%%%%%%%%%%%%%%%%%%%%
    %%%% Ejercicio 1.2
    %%%%%%%%%%%%%%%%%%%%%%%%%%%%%%%%%%%%%%%%%%%%%%
    %%%%%%%%%%%%%%%%%%%%%%%%%%%%%%%%%%%%%%%%%%%%%%%%%%%%%
    \item $\{ 1 + 2x + x^{2},\, 3 + x^{2},\, x + x^{2}\}$

    Sabemos que $\dim(P_{n}) = n + 1$, por lo que $\dim (P_{2}(\mathbb{R})) = 3$. Y por un teorema 
    \textit{Si $V$ es un espacio vectorial de dimensión $n$ y si $S$ es un conjunto en $V$
    con exactamente $n$ vectores, entonces $S$ es una base para $V$ si $S$ genera a $V$ o si $S$ 
    es linealmente independiente}\\
    Por lo tanto, el conjunto $\{ 1 + 2x + x^{2},\, 3 + x^{2},\, x + x^{2}\}$ es base para
    $P_{2}(\mathbb{R})$ si contiene tres vectores linealmente independientes.

    Comprobemos la independencia lineal del conjunto dado. El conjunto de polinomios es
    linealmente independiente si la única combinación lineal que da como resultado el polinomio
    nulo (cero) es la combinación en la que todos los coeficientes son cero. En otras palabras, los
    polinomios son linealmente independientes si la ecuación:
    \begin{equation*}
        \alpha(1 + 2x + x^2) + \beta(3 + x^2) + \gamma(x + x^2) = 0 + 0x + 0x^2 = 0
    \end{equation*}
    donde $\alpha$, $\beta$ y $\gamma$ son escalares; solo tiene solución trivial, es decir, $\alpha = \beta = \gamma = 0$.

    Desarrollamos: $\alpha(1 + 2x + x^2) + \beta(3 + x^2) + \gamma(x + x^2) = 0 + 0x + 0x^2 = 0$ y tenemos que:
    \begin{align*}
        \alpha + 2\alpha x + \alpha x^2 + 3\beta + \beta x^2 + \gamma x + \gamma x^2 &= 0 + 0x + 0x^2 = 0 \\
        (\alpha + 3\beta) + (2\alpha x + \gamma x) + \alpha x^2 + \beta x^2 + \gamma x^2 &= 0 + 0x + 0x^2 = 0 \\
        (\alpha + 3\beta) + (2\alpha + \gamma) x + (\alpha + \beta + \gamma) x^2 &= 0 + 0x + 0x^2 = 0
    \end{align*}
        
    Lo que nos da el siguiente sistema de ecuaciones:
    \begin{align*}
        \alpha + 3\beta &= 0 \\
        2\alpha + \gamma &= 0 \\
        \alpha + \beta + \gamma &= 0
    \end{align*}
    Y resolvemos:
    \begin{align*}
        \alpha + 3\beta &= 0 \implies \alpha = -3\beta \\
        2\alpha + \gamma &= 0 \implies \gamma = -2\alpha = -2(-3\beta) \\
        \alpha + \beta + \gamma &= 0 \implies -3\beta + \beta + (-2(-3\beta)) = 0 \implies -3\beta + \beta + 6\beta = 0 \implies 4\beta = 0 \implies \beta = 0
    \end{align*}

    Como $\beta = 0$, sustituimos $\beta$ en: $\alpha + 3\beta = 0 \implies \alpha + 3(0) = 0 \implies \alpha = 0$

    Sustituimos $\alpha$ en: $2\alpha + \gamma = 0 \implies \gamma = 0$

    Por lo tanto: $\alpha = \beta = \gamma = 0$

    Como $\alpha = \beta = \gamma = 0$, podemos decir que es un conjunto linealmente independiente de 3 vectores.

    Dado que estos polinomios son linealmente independientes y forman un conjunto de 3 polinomios en $P_{2}(\mathbb{R})$ y $P_{2}(\mathbb{R})$ es un espacio con dimensión 3 Entonces por el teorema
    anteriormente citado tenemos que estos polinomios generan $P_{2}(\mathbb{R})$. Y el conjunto es una base
    para $P_{2}(\mathbb{R})$

\end{enumerate}