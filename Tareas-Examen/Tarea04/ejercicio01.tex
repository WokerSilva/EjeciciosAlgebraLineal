\section*{Ejercicio 1}

Sea $T: \mathbb{R}^{3} \rightarrow M_{2\times2}(\mathbb{R})$ 
una función tal que $T(a,b,c) = \left(
                                    \begin{array}{cc}
                                        a & b \\
                                        c & a+b
                                    \end{array}
                                \right)$ Realice lo siguiente
\begin{enumerate}
    \item Demuestre que $T$ es una transformación lineal 
    
    Para demostrar que T es una transformación lineal,
    debemos verificar que cumple las dos condiciones 
    siguientes:\\

    \textbf{Condición de additividad:} $T(a,b,c)+T(d,e,f)=T((a,b,c)+(d,e,f))$

    $
    \begin{aligned}
        T(a,b,c) + T(d,e,f) &=  \left(
                                \begin{array}{cc}
                                    a & b \\
                                    c & a+b
                                \end{array}
                                \right) +
                                \left(
                                \begin{array}{cc}
                                    d & e \\
                                    f & d+e
                                \end{array}
                                \right) \\
                                &= \left(
                                \begin{array}{cc}
                                    a+d & b+e \\
                                    c+f & a+b+d+e
                                \end{array}
                                \right) \\
        &= T((a,b,c)+(d,e,f))
    \end{aligned}
    $\\

    \textbf{Condición de homogeneidad:} $T(ka,kb,kc)=kT(a,b,c)$
    
    $
    \begin{aligned}
        T(ka,kb,kc) &= \left(
                                            \begin{array}{cc}
                                                ka & kb \\
                                                kc & ka+kb
                                            \end{array}
                                        \right) \\
        &= k \left(
                                            \begin{array}{cc}
                                                a & b \\
                                                c & a+b
                                            \end{array}
                                        \right) \\
        &= kT(a,b,c)
    \end{aligned}
    $
       
    $\therefore$ Por lo tanto, $T$ es una transformación lineal.

    \item Verifique el \textit{Teorema de la Dimensión}

    El dominio de $T$ es $\mathbb{R}^{3}$, que tiene dimensión 3.
    El rango de $T$ es el conjunto de matrices $2\times2$ que se pueden obtener como imagen de algún vector en $\mathbb{R}^{3}$.

    Para demostrar que el rango de $T$ es 3, debemos encontrar tres vectores $v_{1}, v_{2}, v_{3}$ en $\mathbb{R}^{3}$ tales que $T(v_{1}), T(v_{2}), T(v_{3})$ sean linealmente independientes. Tomamos los vectores $v_{1} = (1,0,0)$, $v_{2} = (0,1,0)$ y $v_{3} = (0,0,1)$.

\begin{minipage}{0.3\textwidth}
\[
\begin{aligned}
T(v_1) &= \left(
\begin{array}{cc}
1 & 0 \\
0 & 1
\end{array}
\right), \\
T(v_2) &= \left(
\begin{array}{cc}
0 & 1 \\
0 & 1
\end{array}
\right), \\
T(v_3) &= \left(
\begin{array}{cc}
0 & 0 \\
1 & 0
\end{array}
\right).
\end{aligned}
\]    
\end{minipage}
\hspace{15mm}
\begin{minipage}{0.5\textwidth}
Estos vectores son linealmente independientes, ya que si $a(1,0,0) + b(0,1,0) + c(0,0,1) = (0,0,0)$, entonces $a = b = c = 0$. Además, podemos observar que las matrices seleccionadas no generan el dominio de la transformación, ya que no es posible representar cualquier vector en $\mathbb{R}^{3}$ como una combinación lineal de estas matrices. Por lo tanto, la nulidad de la transformación es 0, y el rango es 3, lo que verifica el Teorema de la Dimensión.        
\end{minipage}

\end{enumerate}