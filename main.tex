\documentclass[a4paper,12pt]{article} 
\usepackage[utf8]{inputenc} % Acentos válidos sin problemas
\usepackage[spanish]{babel} % Idioma
\usepackage[style=apa]{biblatex}
\usepackage{csquotes}
%\setlegtth{\parindent}{2px}
%\phantom{abc}

\input{packet}

\begin{document}%----------------------INICIO DOCUMENTO------------|
%------------------------------------------------------------------|
\pagecolor{black}
\color{white}

\thispagestyle{firstpage} % Aplicar estilo de primera página
\noindent
%%%%%%%%%%%%%%%%%%%%%%%%%%%%%%%%%%%%%%%%%%%%%%%%%%%%%%%%%%%%%%%%%%%%%%%%%%%%%%
\large\textbf{Facultad de Ciencias}  \\
Algebra Lineal                \hfill semestre: 2024-1 \\
Silva Huerta Marco            \hfill  \\ % No.Cuenta: 316205326 \\
23 de Agosto del 2023         \hfill \textbf{Guía}    \\
\noindent\rule{7.3in}{2.8pt}
%%%%%%%%%%%%%%%%%%%%%%%%%%%%%%%%%%%%%%%%%%%%%%%%%%%%%%%%%%%%%%%%%%%%%%%%%%%%%%

\vspace{5mm}
\begin{center}
    Ejecicios de la clase de Algebra Lineal: Espacios Vectoriales
\end{center}

\begin{enumerate}
    \item Demostrar que en cualquier espacio vectorial $V$, se cumple que $(a \oplus b) (x + y) = ax + ay + bx + by$ para cuales quiera $x,y \in V$ y $a, b \in F$
    \item Sean $V$ un $F-$espacio vectorial, $x \in V$ y $\lambda \in F. $ Demuestra que si $\lambda x = 0_{v} y \lambda \neq 0 $ entonces $x = 0_{v}$
\end{enumerate}



%-----------------------------------------------------------------|
\end{document}%----------------------FIN DEL DOCUMENTO------------| 