\section*{Ejercicio 1}

Sean $X$ un conjunto no vacío y $F$ un campo.
Sea $f[X] \coloneqq \{f : f \colon X\to F\}$ el conjunto de todas las funciones que van de $X$ a $F$. Entonces $f[x]$ es un espacio vectorial sobre $F$ con las siguientes operaciones para cualesquiera $x \in X$ y $a \in F$:

$$(f+g)(x) = f(x) + g(x)$$

$$(af)(x) = a f(x)$$


Demuéstralo o escribe cuál es la propiedad que no cumple y justifica tu respuesta

\begin{enumerate}
    \item Cerradura bajo la suma: Para demostrar que la suma de dos elementos en f[x] sigue estando en f[x], tomemos dos funciones arbitrarias f y g en f[x]. Entonces, para cualquier x en X, tenemos:
    $$ (f + g)(x) = f(x) + g(x) $$
    Dado que $f(x)$ y $g(x)$ están en $F$ (porque son funciones de $X$ a $F$), su suma también estará en $F$. Por lo tanto, $(f + g)(x)$ está en F para todo x, lo que significa que la suma $f + g$ es una función de $X$ a $F$ y, por lo tanto, pertenece a $f[x]$. La cerradura bajo la suma se cumple.    
\end{enumerate}



Cerradura bajo la multiplicación por escalar: Ahora, tomemos una función f en f[x] y un escalar a en F. La multiplicación por escalar se define como:

(af)(x) = af(x)

Dado que f(x) está en F y a también está en F, el producto af(x) también estará en F. Esto significa que (af)(x) está en F para todo x, y por lo tanto, la función af(x) pertenece a f[x]. La cerradura bajo la multiplicación por escalar se cumple.

Asociatividad de la suma: La propiedad de asociatividad de la suma en f[x] sigue directamente de la asociatividad de la suma en F, ya que estamos sumando funciones de X a F y, en cada punto x, estamos sumando elementos de F.

Conmutatividad de la suma: Similarmente, la conmutatividad de la suma en f[x] sigue de la conmutatividad de la suma en F, ya que estamos sumando elementos de F en cada punto x.

Elemento neutro de la suma: La función nula o cero, que mapea todos los elementos de X a 0 en F, actúa como el elemento neutro de la suma en f[x].

Inverso aditivo: Dada una función f en f[x], su inverso aditivo sería la función -f(x), que mapea cada x a -f(x) en F. Esta función existe en f[x] debido a la estructura del campo F.

Todas las propiedades necesarias para que f[x] sea un espacio vectorial sobre F están satisfechas. Por lo tanto, las operaciones definidas cumplen con las condiciones requeridas para formar un espacio vectorial de funciones.
