\section*{Ejercicio 1}

Sean $T : \R^{2} \rightarrow P_{1}(\R)$ una transformación lineal. Sean $\beta = \{ (2,3), (-1,4) \}$ y $\gamma = \{ 3 + 2x, 4\}$
bases ordenadas de los respectivos espacios vectoriales. Si

\begin{equation}
    \left[ T \right]_{\beta}^{\gamma}
     \begin{pmatrix}
        48 & 48 \\
       -23 & -19 \\
      \end{pmatrix}  
\end{equation}

\begin{enumerate}
%%%%%%%%%%%%%%%%%%%%%%%%%%%%%%%%%%%%
%%%%%%%%%%%%%%%%%%%%%%%%%%%%%%%%%%%%
%%%% Ejercicio 1.1    
    \item Hallar $T(3,-1)$
%%%%%%%%%%%%%%%%%%%%%%%%%%%%%%%%%%%%

    Se tiene que, por la observación $1.39$ de la matriz de representación matricial:
    \begin{equation*}
        [T(2,3)]_{\gamma}= 1/4
            \begin{pmatrix}
            48\\
            -23
        \end{pmatrix}
    \end{equation*}
    \begin{equation*}
        [T((-1,4))]_{\gamma}= 1/4
        \begin{pmatrix}
            48\\
            -19
        \end{pmatrix}
    \end{equation*}
    Dado a que $T$ es lineal, se cumple que 
    \begin{itemize}
        \item $-1T(a)=T(-a)$, para algún a $\in \R^2$
        \item $T(a) + T(b) = T(a+b)$, para a,b $\in \R^2$
    \end{itemize}
        Por lo tanto, tenemos que $T((2,3))-1T((-1,4))=T((2,3)-(-1,4))=T((3,-1))$. Para obtener el valor de $T((3,-1))$, veamos que:
    \begin{equation*}
        [T((3,-1))]_\gamma=[T((2,3)-(-1,4))]_{\gamma}=[T(2,3)]_{\gamma}-[T((-1,4))]_{\gamma}
    \end{equation*}
    \begin{equation*}
        = 1/4\begin{pmatrix}
            48\\
            -23
        \end{pmatrix}
        -1/4 \begin{pmatrix}
            48\\
            -19
        \end{pmatrix}
    \end{equation*}
    \begin{equation*}
        1/4 \begin{pmatrix}
        48-48\\
        -23-19
        \end{pmatrix}
        =1/4\begin{pmatrix}
        0\\
        -4
        \end{pmatrix}
        =\begin{pmatrix}
        0\\
        -1
        \end{pmatrix}
    \end{equation*}
    Por lo que $T((3,-1)) = (0)(3 + 2x) + (-1)(4) = -4$\\

%%%%%%%%%%%%%%%%%%%%%%%%%%%%%%%%%%%%
%%%%%%%%%%%%%%%%%%%%%%%%%%%%%%%%%%%%
%%%% Ejercicio 1.2 
    \item Encuentre la regla de correspondencia de dicha transformación lineal. 
%%%%%%%%%%%%%%%%%%%%%%%%%%%%%%%%%%%%

    Por el Teorema $1.54$, para cualquier v=(a,b)$\in \R^2$, se tiene que:
    \begin{equation*}
        [T(v)]_\gamma = [T]_\gamma^\beta [v]_\beta
    \end{equation*}
    \begin{equation*}
        \begin{pmatrix}
            a\\
            b
        \end{pmatrix}
        =\alpha \begin{pmatrix}
            2\\
            3
        \end{pmatrix}
        + \beta 
        \begin{pmatrix}
            -1\\
            4
        \end{pmatrix}
    \end{equation*}
    Tenemos que resolver el siguiente sistema de ecuaciones, para encontrar a $[v]_\beta$:
    \begin{equation*}
        2\alpha-\beta=a
    \end{equation*}
    \begin{equation*}
        3\alpha+4\beta=b
    \end{equation*}
    \begin{equation*}
        \beta=2\alpha - a
    \end{equation*}
    \begin{equation*}
        3\alpha+4(2\alpha - a)=b
    \end{equation*}
    \begin{equation*}
        11\alpha - 4a=b
    \end{equation*}
    \begin{equation*}
        \alpha=\frac{4a+b}{11}
    \end{equation*}
    \begin{equation*}
        \beta=2(\frac{4a+b}{11}) - a
    \end{equation*}
    \begin{equation*}
        \beta=\frac{8a+2b}{11}-a
    \end{equation*}
    \begin{equation*}
        \beta=\frac{-3a+2b}{11}
    \end{equation*}

    \begin{equation*}
        [v]_\beta= \begin{pmatrix}
        \frac{4a+b}{11}\\
        \frac{-3a+2b}{11}
        \end{pmatrix}
        = 1/11\begin{pmatrix}
            4a+b\\
            -3a+2b
        \end{pmatrix}
    \end{equation*}
    Ya que tenemos el vector columna de v en $\gamma$, lo multiplicamos por la matriz de representación de $\beta$ a $\gamma$:
    \begin{equation*}
        [T]_\gamma^\beta [v]_\beta = 1/4 \begin{pmatrix}
            48 & 48\\
            -23 & -19
        \end{pmatrix}
        1/11\begin{pmatrix}
            4a+b\\
            -3a+2b
        \end{pmatrix}
    \end{equation*}
    \begin{equation*}
        [T]_\gamma^\beta [v]_\beta=1/44
        \begin{pmatrix}
            48 & 48\\
            -23 & -19
        \end{pmatrix}
        \begin{pmatrix}
           4a+b\\
            -3a+2b
        \end{pmatrix}
    \end{equation*}
    \begin{equation*}
        [T]_\gamma^\beta [v]_\beta=1/44\begin{pmatrix}
        48a+144b \\
        -35a-61b
        \end{pmatrix}
    \end{equation*}
    Ya que tenemos esto, multiplicamos por los vectores de la base $\gamma$:\\
    \begin{equation*}
        1/44(48a+144b)(3+2x)=\frac{(12)(a+3b)(3+2x)}{11}
    \end{equation*}
    \begin{equation*}
        1/44(-35a-61b)(4)=\frac{-35a-61b}{11} 
    \end{equation*}
    Entonces se tiene que:
    \begin{equation*}
        T((a,b))=\frac{(12)(a+3b)(3+2x)}{11}+\frac{-35a-61b}{11}=\frac{(12)(a+3b)(3+2x)-35a-61b}{11}
    \end{equation*}

%%%%%%%%%%%%%%%%%%%%%%%%%%%%%%%%%%%%
%%%%%%%%%%%%%%%%%%%%%%%%%%%%%%%%%%%%
%%%% Ejercicio 1.1        
    \item Determine el núcleo y la imagen de bases para cada uno de estos subespacios.
%%%%%%%%%%%%%%%%%%%%%%%%%%%%%%%%%%%%

    De la regla de correspondencia, se tiene que, para que T(a,b)=0, se debe cumplir lo siguiente:
    $(12)(a + 3b)(3 + 2x) = 0$, pues es el Único con termino de 1er grado, y $-35a - 61b = 0$. Se tiene entonces que:
    \begin{equation*}
        (12)(a+3b)(3+2x)=0 \longrightarrow a+3b=0 \longrightarrow a=-3b
    \end{equation*}
    \begin{equation*}
        -35(-3b)-61b=0
    \end{equation*}
    \begin{equation*}
        44b=0 \longrightarrow b=0
    \end{equation*}
    \begin{equation*}
        a+3(0)=0 \longrightarrow a=0
    \end{equation*}
    Por lo tanto, para que la transformación dé el vector $0$, $v$ tiene que ser el vector $(0,0)$ 
     en $\R^2$, por lo tanto $Ker(T) = \{ 0 \}$, cuya base es $\emptyset$. Dado que el
     $Ker(T) = \{0\}$, se tiene que $T$ es inyectiva, y ya que $dim(\R^2) = dim(P_{1}(\R))$,
     y por el teorema $1.25$, $T$ es suprayectiva, por lo tanto, su imagen es $P_1(\R)$, 
     cuya base propuesta es $\Theta = \{x,1\}$

    
\end{enumerate}