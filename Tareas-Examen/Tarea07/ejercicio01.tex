\section*{Ejercicio 1}

\noindent Considérense las siguientes bases de $\mathbb{R}^{3}$
\begin{equation*}
    \beta = \{ (1,0,0),(0,1,0),(0,0,1) \}
\end{equation*}
\begin{equation*}
    \gamma = \{ (1,0,1),(2,1,2),(1,2,2) \}
\end{equation*}

\noindent Y la transformación $T: \Rtres \rightarrow \Rtres$ tal que $T(x,y,z) = (2x,x+y,y+z)$ 
\begin{enumerate}
    %%%--- #########################################################################################
    %%%--- #########################################################################################
    \item Encuentra la matriz de cambio de coordenadas de la base $\beta$ y $\gamma$
    %%%--- #########################################################################################
    %%%--- #########################################################################################

    \vspace{2mm}

    \noindent \solucion

    \begin{center}
        Escribimos los vectores de $\beta$ como columnas de una matriz

        \vspace{5mm}

        $
        \beta = 
            \begin{pmatrix}
                1 & 0 & 0 \\
                0 & 1 & 0 \\
                0 & 0 & 1
            \end{pmatrix}
        $
        \vspace{5mm}

        Multiplicamos esta matriz por $P$

        \vspace{5mm}

        $
        P \beta = 
            \begin{pmatrix}
                1 & 2 & 1 \\
                0 & 1 & 2 \\
                0 & 0 & 2
            \end{pmatrix}
            \begin{pmatrix}
                1 & 0 & 0 \\
                0 & 1 & 0 \\
                0 & 0 & 1
            \end{pmatrix} = 
            \begin{pmatrix}
                1 & 2 & 1 \\
                2 & 1 & 2 \\
                2 & 2 & 2
            \end{pmatrix}
        $

        \vspace{5mm}

        El resultado debe ser la matriz que contiene los vectores de $\gamma$ como columnas.

        \vspace{5mm}

        $
        \gamma = 
            \begin{pmatrix}
                1 & 2 & 1 \\
                2 & 1 & 2 \\
                2 & 2 & 2
            \end{pmatrix}
        $
    \end{center}

    \vspace{5mm}

    Por lo tanto, la matriz de cambio de coordenadas es la siguiente:

    \begin{align*}
        P & = 
        \begin{pmatrix}
            1 & 2 & 1 \\
            0 & 1 & 2 \\
            0 & 0 & 2
        \end{pmatrix} \\
    \end{align*}
    \vspace{5mm}
    %%%--- #########################################################################################
    %%%--- #########################################################################################
    \item Encuentra la matriz de cambio de coordenadas de la base $\gamma$ y $\beta$ 
    %%%--- #########################################################################################
    %%%--- #########################################################################################

    \vspace{2mm}

    \noindent \solucion

    \begin{center}
        Escribimos los vectores de $\gamma$ como columnas de una matriz.

        \vspace{5mm}

        $
        \gamma = 
        \begin{pmatrix}
            1 & 0 & 1 \\
            2 & 1 & 2 \\
            1 & 2 & 2
        \end{pmatrix}
        $        

        \vspace{5mm}

        Inversamos esta matriz.

        \vspace{5mm}

        $
        \gamma^{-1} = 
            \begin{pmatrix}
                1 & 0 & -\frac{1}{2} \\
                0 & 1 & -\frac{1}{2} \\
                0 & 0 & \frac{1}{2}
            \end{pmatrix}
        $

        \vspace{5mm}
        
        Multiplicamos esta matriz por $\gamma$

        \vspace{5mm}

        $
        Q \gamma = 
            \begin{pmatrix}
                1 & 0 & -\frac{1}{2} \\
                0 & 1 & -\frac{1}{2} \\
                0 & 0 & \frac{1}{2}
            \end{pmatrix}
            \begin{pmatrix}
                1 & 0 & 1 \\
                2 & 1 & 2 \\
                1 & 2 & 2
            \end{pmatrix} = 
            \begin{pmatrix}
                1 & 0 & 0 \\
                0 & 1 & 0 \\
                0 & 0 & 1
            \end{pmatrix}
        $

        \vspace{5mm}    

        El resultado debe ser la matriz que contiene los vectores de $\beta$
        
        \vspace{5mm}    
        \begin{align*}
            Q & = 
            \begin{pmatrix}
                1 & 0 & -\frac{1}{2} \\
                0 & 1 & -\frac{1}{2} \\
                0 & 0 & \frac{1}{2}
            \end{pmatrix} \\
        \end{align*}
    
    \end{center}
    

    %%%--- #########################################################################################
    %%%--- #########################################################################################
    \item Comprueba que $\left[ T \right]_{\gamma} = Q^{-1} \left[ T \right]_{\beta} Q$ 
    %%%--- #########################################################################################
    %%%--- #########################################################################################

    \noindent \solucion

    \textbf{Matriz de $T$ en la base $\beta$}
    
    \vspace{2mm}    

    La matriz de $T$ en la base $\beta$ es la matriz que representa el operador lineal $T$ cuando se expresan los
    vectores en la base $\beta$. Esta matriz se puede encontrar multiplicando la matriz de $T$ por la matriz de
    cambio de coordenadas de $\gamma$ a $\beta$.

    \begin{equation*}
        [T]_{\beta} = P^{-1} [T] P
    \end{equation*}

    Donde $P$ es la matriz de cambio de coordenadas de $\gamma$ a $\beta$ y $[T]$ es la matriz de $T$ en la base
    estándar. Sabemos que $P$ es la siguiente matriz:

    \begin{center}
        $
        P = 
            \begin{pmatrix}
                1 & 2 & 1 \\
                0 & 1 & 2 \\
                0 & 0 & 2
            \end{pmatrix}
        $
    \end{center}

    Y sabemos que $[T]$ es la siguiente matriz:

    \begin{center}
        $
        [T] = 
            \begin{pmatrix}
                2 & 1 & 0 \\
                1 & 1 & 1 \\
                0 & 0 & 1
            \end{pmatrix}
        $
    \end{center}

    Por lo tanto, la matriz de $T$ en la base $\beta$ es la siguiente:

    \begin{center}
        $
        [T]_{\beta} = P^{-1} [T] P = 
            \begin{pmatrix}
                1 & -\frac{1}{2} & 0 \\
                0 & 0 & \frac{1}{2} \\
                0 & 0 & \frac{1}{2}
            \end{pmatrix}    
        $
    \end{center}
    
    %%%%%%%%%%%%%%%%%%%%%%%%%%%%%%%%%%%%%%%%%%%%%%%%%%%%%%%%%%%%%%%%%%%%%%%%%%%%%%%%%
    %%%%%%%%%%%%%%%%%%%%%%%%%%%%%%%%%%%%%%%%%%%%%%%%%%%%%%%%%%%%%%%%%%%%%%%%%%%%%%%%%
    %%%%%%%%%%%%%%%%%%%%%%%%%%%%%%%%%%%%%%%%%%%%%%%%%%%%%%%%%%%%%%%%%%%%%%%%%%%%%%%%%
    %%%%%%%%%%%%%%%%%%%%%%%%%%%%%%%%%%%%%%%%%%%%%%%%%%%%%%%%%%%%%%%%%%%%%%%%%%%%%%%%%

    \textbf{Matriz de $T$ en la base $\gamma$}

    \vspace{2mm}    

    La matriz de $T$ en la base $\gamma$ es la matriz que representa el operador lineal $T$ cuando se
    expresan los vectores en la base $\gamma$. Esta matriz se puede encontrar multiplicando la matriz
    de $T$ por la matriz de cambio de coordenadas de $\beta$ a $\gamma$.

    \begin{equation*}
        [T]_{\gamma} = Q [T] Q^{-1}
    \end{equation*}

    Donde $Q$ es la matriz de cambio de coordenadas de $\beta$ a $\gamma$ y $[T]$ es la matriz de $T$
    en la base estándar. Sabemos que $Q$ es la siguiente matriz:

    \begin{center}
        $
        Q = 
            \begin{pmatrix}
                1 & 0 & -\frac{1}{2} \\
                0 & 1 & -\frac{1}{2} \\
                0 & 0 & \frac{1}{2}
            \end{pmatrix}
        $
    \end{center}

    Y sabemos que $[T]$ es la siguiente matriz:

    \begin{center}
        $
        [T] = 
            \begin{pmatrix}
                2 & 1 & 0 \\
                1 & 1 & 1 \\
                0 & 0 & 1
            \end{pmatrix}
        $
    \end{center}

    Por lo tanto, la matriz de $T$ en la base $\gamma$ es la siguiente:

    \begin{center}
        $
        [T]_{\gamma} = Q [T] Q^{-1} = 
            \begin{pmatrix}
                2 & 0 & 1 \\
                0 & 1 & 1 \\
                1 & 0 & 1
            \end{pmatrix}
        $    
    \end{center}

    Ahora podemos comprobar si la ecuación $\left[ T \right]_{\gamma} = Q^{-1} \left[ T \right]_{\beta} Q$ 

    \begin{center}
        $
        Q^{-1} \left[ T \right]_{\beta} Q = 
            \begin{pmatrix}
                1 & 0 & \frac{1}{2} \\
                0 & 1 & \frac{1}{2} \\
                0 & 0 & -\frac{1}{2}
            \end{pmatrix}
            \begin{pmatrix}
                1 & -\frac{1}{2} & 0 \\
                0 & 0 & \frac{1}{2} \\
                0 & 0 & \frac{1}{2}
            \end{pmatrix}
            \begin{pmatrix}
                1 & 0 & -\frac{1}{2} \\
                0 & 1 & -\frac{1}{2} \\
                0 & 0 & \frac{1}{2}
            \end{pmatrix} = 
            \begin{pmatrix}
                2 & 0 & 1 \\
                0 & 1 & 1 \\
                1 & 0 & 1
            \end{pmatrix}
        $
    \end{center}


    La ecuación $\left[ T \right]_{\gamma} = Q^{-1} \left[ T \right]_{\beta} Q$ se cumple. Esto significa que
    la matriz de $T$ en la base $\gamma$ se puede obtener multiplicando la matriz de $T$ en la base $\beta$
    por la matriz de cambio de coordenadas de $\beta$ a $\gamma$, y luego multiplicando por la inversa de
    la matriz de cambio de coordenadas.
\end{enumerate}