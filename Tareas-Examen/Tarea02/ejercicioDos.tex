\section*{Ejercicio 2}

En cada uno de los siguientes incisos, determine (y demuestre, si es el caso) si el conjunto es linealmente
independiente o no.

\begin{itemize}
    \item[(a)] En $P_{2}(\mathbb{R})$ el conjunto $S = \{ 1, 1 - x + 2x^{2}, 2 + 3x - x^{2} \}$  

    Para comprobar si el conjunto \(S = \{1, 1 - x + 2x^{2}, 2 + 3x - x^{2}\}\) en \(P_{2}(\mathbb{R})\) 
    es linealmente independiente, primero,  vemos la definición de \textit{linealmente independiente},
    donde para un conjunto de polinomios 
    \(\{p_1(x), p_2(x), p_3(x), \ldots, p_n(x)\}\)
    se considera linealmente independiente en \(P_{n}(\mathbb{R})\)
    si la única forma de escribir la combinación lineal 
    igual a cero es haciendo que todos los coeficientes
    sean iguales a cero:

    $$ c_{1} p_{1} (x) + c_{2}p_{2}(x) + c_{3} p_{3} (x) + \cdots + c_{n} p_{n}(x)$$

    donde \(c_1, c_2, c_3, \ldots, c_n\) son coeficientes reales y \(p_i(x)\) son los polinomios en el conjunto.

    Ahora la combinación lineal la igualamos a cero. 

    \[
        c_1(1) + c_2(1 - x + 2x^{2}) + c_3(2 + 3x - x^{2}) = 0
    \]

    Donde \(c_1, c_2, c_3\) son los coeficientes que queremos 
    encontrar. Distribuimos los coeficientes en la ecuación
    y agrupamos términos semejantes:

    \[
        c_1 + c_2(1 - x + 2x^{2}) + c_3(2 + 3x - x^{2}) = 0
    \]

    Obtenemos:    
    \begin{align*}
        c_1 + c_2 - c_2x + 2c_2x^{2} + 2c_3 + 3c_3x - c_3x^{2} & = 0 \\
        (c_1 + c_2 + 2c_3) + (-c_2 - c_3)x + (2c_2 - c_3)x^{2} & = 0
    \end{align*}
    

    Entonces la ecuación debe ser igual a cero para todo \(x\), y como esta ecuación debe ser verdadera para todos los valores de \(x\), los coeficientes de cada término deben ser igual a cero. Esto nos lleva a un sistema de ecuaciones:

    \begin{align*}
        1. & \quad c_1 + c_2 + 2c_3 = 0 \\
        2. & \quad -c_2 - c_3 = 0 \\
        3. & \quad 2c_2 - c_3 = 0
    \end{align*}

    De la ecuación 2, tenemos:
    \begin{align*}
        -c_2 - c_3 = 0 & \\        
        \implies &  c_3 = -c_2
    \end{align*}

    Sustituimos este valor en la ecuación 3:
    \begin{align*}
        2c_2 - c_3    & = 0  \\
        2c_2 - (-c_2) & = 0  \\
        2c_2 + c_2    & = 0  \\
        3c_2          & = 0  \\
    \end{align*}


    Por lo tanto, \(c_2 = 0\). Ahora, sustituimos \(c_2 = 0\) en la ecuación 1:
    \begin{align*}
        c_1 + c_2 + 2c_3 = 0  & = 0  \\
        c_1 + 0 + 2(-c_2) = 0 & = 0  \\
        c_1 - 2c_2 = 0        & = 0  \\
        c_1 - 2(0) = 0        & = 0  \\
        c_1                   & = 0
    \end{align*}

    Hemos encontrado que \(c_1 = 0\), \(c_2 = 0\), y \(c_3 = 0\),
    lo que significa que la única forma de obtener la combinación
    lineal igual a cero es haciendo que todos los coeficientes 
    sean iguales a cero. Por lo tanto, el conjunto
     \(S = \{1, 1 - x + 2x^{2}, 2 + 3x - x^{2}\}\) 
     es linealmente independiente en \(P_{2}(\mathbb{R})\).

    %%%%%%%%%%%%%%%%%%%%%%%%%%%%%%%%%%%%%%%%%%%%%%%%%%%%%%%%%%%%%%%%%%%%%%%%
    \item[(b)] En $ M_{3 \times 3}$, el conjunto
    \begin{equation}
        S = \left\{
            \begin{pmatrix}
                6 & 2 \\
                -5 & 0 \\
            \end{pmatrix},         
            \begin{pmatrix}
                0 & 3 \\
                -4 & -5 \\
            \end{pmatrix},        
            \begin{pmatrix}
                3 & 0 \\
                -2 & 5 \\
            \end{pmatrix}
            \right\}
    \end{equation}

    Para determinar si el conjunto $S$ es linealmente independiente
    empezamos por verificar si la única combinación lineal 
    que iguala el vector nulo $\mathbf{0}$ es
    la combinación lineal en la que todos los coeficientes son cero.
    \[
        c_1 \mathbf{v}_1 + c_2 \mathbf{v}_2 + c_3 \mathbf{v}_3 = \mathbf{0},
    \]

    donde $c_1$, $c_2$, y $c_3$ son coeficientes escalares
    y $\mathbf{v}_1$, $\mathbf{v}_2$, y $\mathbf{v}_3$
    son los vectores del conjunto $S$, así que:    

    \[
        c_1
        \begin{pmatrix}
            6 & 2 \\
            -5 & 0 \\
        \end{pmatrix}
        +
        c_2
        \begin{pmatrix}
            0 & 3 \\
            -4 & -5 \\
        \end{pmatrix}
        +
        c_3
        \begin{pmatrix}
            3 & 0 \\
            -2 & 5 \\
        \end{pmatrix}
        =
        \begin{pmatrix}
            0 & 0 \\
            0 & 0 \\
        \end{pmatrix}
    \]

    Ahora, podemos escribir dos ecuaciones lineales a
    partir de esta igualdad. Para la primera fila de la matriz,
    tenemos:
        
    \[
        6c_1 + 0c_2 + 3c_3 = 0 \quad (1)
    \]

    Y para la segunda fila:

    \[
        2c_1 + 3c_2 + 0c_3 = 0 \quad (2)
    \]

    Resolvamos este sistema de ecuaciones para encontrar
    los valores de $c_1$, $c_2$, y $c_3$ que hacen que la
    igualdad sea verdadera.

    Comencemos con la ecuación (1):

    \begin{align*}
        6c_1 + 0c_2 + 3c_3 & = 0\\
        6c_1 + 3c_3        & = 0
    \end{align*}

    Dividimos ambos lados por 3:

    \[
        2c_1 + c_3 = 0
    \]

    Ahora, vamos a la ecuación (2):
    \begin{align*}
        2c_1 + 3c_2 + 0c_3 & = 0 \\
        2c_1 + 3c_2        & = 0
    \end{align*}

    Ahora tenemos un sistema de dos ecuaciones con tres incógnitas:

    \begin{align*}
        2c_1  + c_3 &= 0 \quad &(1) \\
        2c_1 + 3c_2 &= 0 \quad &(2)
    \end{align*}


    Para determinar si el conjunto es linealmente independiente o no, podemos usar el método de reducción por fila. Primero, restamos la ecuación (1) de la ecuación (2):
    
    \begin{align*}
        (2) - (1): & \\
        & (2c_1 + 3c_2) - (2c_1 + c_3) = 0 - 0
    \end{align*}

    Esto simplifica a:

    \[
        2c_1 + 3c_2 - 2c_1 - c_3 = 0
    \]

    Los términos $2c_1$ y $-2c_1$ se cancelan

    \[
        3c_2 - c_3 = 0
    \]

    Se ha simplificado:
    \begin{align*}
        3c_2 - c_3 &= 0 \quad &(3) \\
        2c_1 + c_3 &= 0 \quad &(1)
    \end{align*}

    A partir de la ecuación (3), podemos despejar $c_3$:

    \[
        c_3 = 3c_2
    \]

    Y ahora podemos sustituir esto en la ecuación (1):
    \begin{align*}
        2c_1 + 3c_2 & = 0 \\
        2c_1 + 9c_2 & = 0
    \end{align*}
    
    Dividimos por 2:

    \[
        c_1 + 4.5c_2 = 0
    \]

    Ahora tenemos un sistema en términos de $c_1$ y $c_2$:

    \begin{align*}
        c_1 + 4.5c_2 &= 0 \quad &(4) \\
        3c_2 - c_3 &= 0 \quad &(3)
    \end{align*}


    Para que el conjunto sea linealmente independiente, todos los coeficientes $c_1$, $c_2$, y $c_3$ deben ser iguales a cero. Sin embargo, podemos ver que las ecuaciones (3) y (4) son inconsistentes. Por ejemplo, si tomamos $c_2 = 1$, entonces la ecuación (3) nos dice que $c_3 = 3$, pero la ecuación (4) nos dice que $c_1 = -4.5$, lo cual no es posible.
    Dado que no podemos encontrar una solución en la que $c_1$, $c_2$, y $c_3$ sean todos iguales a cero, el conjunto $S$ es linealmente dependiente.

\end{itemize}