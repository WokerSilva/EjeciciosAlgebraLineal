\section*{Ejercicio 3}

\begin{enumerate}
    \item Encuentra la base dual de $\Rdos, \beta = \{ (1,2),(3,4) \}$
    
    \noindent \solucion \\

    La base dual de un espacio vectorial es un conjunto de formas lineales
    (funciones lineales que toman vectores y devuelven escalares) que 
    actúan sobre los vectores de la base original. Para una base $\beta = \{v_{1},v_{2}\}$
    en $\Rdos$, la base dual $\beta^{*} = \{ f_{1},f_{2} \}$ se define tal que
    $f_{i}(v_{j}) = \delta_{ij}$, donde $\delta_{ij}$ es la delta de Kronecker que es 1 si
    $i = j$ y 0 en caso contrario.

    Para nuestra base $\beta = \{ (1,2),(3,4) \}$ podemos encontrar la base dual
    resolviendo el sistema de ecuaciones lineales para $f_{i}(v_{j}) = \delta_{ij}$.

    \begin{align*}
        f_1((1,2)) & = 1 \\
        f_1((3,4)) & = 0 \\
        f_2((1,2)) & = 0 \\
        f_2((3,4)) & = 1 \\
    \end{align*}

    Resolviendo este sistema, obtenemos la base dual 
    \begin{equation*}
        \beta^{*} = \{ (2,-1), \left( \frac{-3}{2} , 1 \right) \}
    \end{equation*}

    \item Encuentra la base $\beta$ de $V = P_{1} (\R)$ cuya base dual es
          $\beta^{*} = \{ f_{1},f_{2} \}$, siendo 
          $f_{1} \left[ p(x) \right] = \int_{0}^{1} p(x) \, dx$
          y $f_{2} \left[ p(x) \right] = \int_{0}^{2} p(x) \, dx$ 

    \noindent \solucion \\

    Necesitamos encontrar dos polinomios $p_{1}(x), p_{2}(x) \in p_{1}(R)$ tal que 
    $f_{i}(p_{j}(x)) = \delta_{ij}$

    Esto significa que necesitamos encontrar $p_{1}(x),p_{2}(x)$ tal que:

    \begin{align*}
        \int_{0}^{1} p_1(x) \, dx &= 1 \\
        \int_{0}^{2} p_1(x) \, dx &= 0 \\
        \int_{0}^{1} p_2(x) \, dx &= 0 \\
        \int_{0}^{2} p_2(x) \, dx &= 1 \\
    \end{align*}

    Entonces:
    \begin{equation*}
        \beta = \{ 1 - 2x, x - \frac{x^{2}}{2}  \}
    \end{equation*}
\end{enumerate}

